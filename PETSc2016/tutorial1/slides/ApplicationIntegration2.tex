\begin{frame}[fragile]{Incorporating PETSc into Existing Codes}
  \begin{block}{PETSc does not seize \lstinline|main()|, does not control output}   \end{block} \vspace*{-0.4cm}
  %\pause
  \begin{block}{Propogates errors from underlying packages, flexible}   \end{block} \vspace*{-0.4cm}
  %\pause
  \begin{block}{Nothing special about \lstinline|MPI_COMM_WORLD|}   \end{block} \vspace*{-0.4cm}
  %\pause
  \begin{block}{Can wrap existing data structures/algorithms}
    \begin{itemize} \vspace*{-0.2cm}
    \item \lstinline|MatShell|, \lstinline|PCShell|, full implementations
    \item \lstinline|VecCreateMPIWithArray()|
    \item \lstinline|MatCreateSeqAIJWithArrays()|
    \item Use an existing semi-implicit solver as a preconditioner
    \item Usually worthwhile to use native PETSc data structures \\
      unless you have a good reason not to
    \end{itemize}
  \end{block} \vspace*{-0.4cm}
  %\pause
  \begin{block}{Uniform interfaces across languages}
    \begin{itemize} \vspace*{-0.2cm}
    \item C, C++, Fortran 77/90, Python, MATLAB
    \end{itemize}
  \end{block} \vspace*{-0.4cm}
  %\pause
  \begin{block}{Do not have to use high level interfaces (e.g.~SNES, TS, DM)}
    \begin{itemize} \vspace*{-0.2cm}
    \item but PETSc can offer more if you do, like MFFD and SNES Test
    \end{itemize}
  \end{block}
\end{frame}
