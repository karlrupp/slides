

%
% Introduction: Motivation for this work
%

\begin{frame}{Outline}
 \begin{center}
  \includegraphics[width=0.9\textwidth]{figures/outline}
 \end{center}
\end{frame}



\begin{frame}{Contents}
  \begin{center}
   \Large Part 0: What is this all about?
  \end{center}
\end{frame}


\begin{frame}{Contents}

 \begin{minipage}{0.5\textwidth}
  \begin{block}{Libraries}
   \begin{itemize}
    \item ViennaCL
    \item ViennaData
    \item ViennaFEM
    \item ViennaGrid
    \item ViennaIPD
    \item ViennaMath
    \item ViennaMesh
   \end{itemize}
  \end{block}
 \end{minipage}
%
 \begin{minipage}{0.45\textwidth}
  \begin{block}{Applications}
   \begin{itemize}
    \item ViennaMOS
    \item ViennaProfiler
    \item ViennaSHE
    \item ViennaWD
    \item ViennaX
    \item
    \item 
   \end{itemize}
  \end{block}
 \end{minipage}
 

\end{frame}




\begin{frame}{Contents}

  \begin{block}{Covered In This Talk}
   \begin{itemize}
    \item Requirements in Computational Science
    \item GPU Computing
    \item Providing High-Level Interfaces
    \item Avoiding Monolithic Code
    \item Abstractions: From Math to Code (and back)
   \end{itemize}
  \end{block}

  \begin{block}{Not Covered}
   \begin{itemize}
    \item C++11
    \item Maximize use of Boost
   \end{itemize}
  \end{block}

\end{frame}




\begin{frame}{The Computational Scientist}
  \begin{block}{A Strange Animal}
   \begin{itemize}
    \item Goal is science, not to execute software
    \item Codes seldomly designed for 'large scale' upfront
    \item Scientists receive software training from scientists
    \item Performance vs. portability and maintainability
    \item Run code on clusters (headless, old software stack, ...)
    \item \emph{Come and go}
   \end{itemize}
  \end{block}

  \vspace*{2cm}
  \begin{block}{}
   \footnotesize Basili \textit{et al.}, Understanding the High-Performance-Computing Community: A Software Engineer's Perspective. IEEE Software, 2008.
  \end{block}

\end{frame}


\begin{frame}{Simulation Flow}
 \begin{block}{The Many Steps in Simulating a FinFET}
  \begin{center}
   \includegraphics[width=0.65\textwidth]{intel-trigate.png} \\
   {\tiny (C) Intel, 2011 }
  \end{center}
 \end{block}
\end{frame}


% 
% 
% \begin{frame}{Simulation Flow}
%  \begin{block}{The Many Steps in Simulating a FinFET}
%   \begin{center}
%    \includegraphics[width=0.95\textwidth]{flow-model.png}
%   \end{center}
%  \end{block}
% \end{frame}
% 
% \begin{frame}{Simulation Flow}
%  \begin{block}{The Many Steps in Simulating a FinFET}
%   \begin{center}
%    \includegraphics[width=0.95\textwidth]{flow-meshing-1.png}
%   \end{center}
%  \end{block}
% \end{frame}
% 
% \begin{frame}{Simulation Flow}
%  \begin{block}{The Many Steps in Simulating a FinFET}
%   \begin{center}
%    \includegraphics[width=0.95\textwidth]{flow-meshing-2.png}
%   \end{center}
%  \end{block}
% \end{frame}
% 
% \begin{frame}{Simulation Flow}
%  \begin{block}{The Many Steps in Simulating a FinFET}
%   \begin{center}
%    \includegraphics[width=0.95\textwidth]{flow-simulate.png}
%   \end{center}
%  \end{block}
% \end{frame}
% 
% 
% \begin{frame}{Simulation Flow}
%  \begin{block}{The Many Steps in Simulating a FinFET}
%   \begin{center}
%    \includegraphics[width=0.95\textwidth]{flow-viennagrid.png}
%   \end{center}
%  \end{block}
% \end{frame}
% 
% \begin{frame}{Simulation Flow}
%  \begin{block}{The Many Steps in Simulating a FinFET}
%   \begin{center}
%    \includegraphics[width=0.95\textwidth]{flow-grid-data.png}
%   \end{center}
%  \end{block}
% \end{frame}
% 
% \begin{frame}{Simulation Flow}
%  \begin{block}{The Many Steps in Simulating a FinFET}
%   \begin{center}
%    \includegraphics[width=0.95\textwidth]{flow-viennadata.png}
%   \end{center}
%  \end{block}
% \end{frame}
% 
% \begin{frame}{Simulation Flow}
%  \begin{block}{The Many Steps in Simulating a FinFET}
%   \begin{center}
%    \includegraphics[width=0.95\textwidth]{flow-poisson.png}
%   \end{center}
%  \end{block}
% \end{frame}
% 
% \begin{frame}{Simulation Flow}
%  \begin{block}{The Many Steps in Simulating a FinFET}
%   \begin{center}
%    \includegraphics[width=0.95\textwidth]{flow-viennamath.png}
%   \end{center}
%  \end{block}
% \end{frame}
% 
% \begin{frame}{Simulation Flow}
%  \begin{block}{The Many Steps in Simulating a FinFET}
%   \begin{center}
%    \includegraphics[width=0.95\textwidth]{flow-viennafem.png}
%   \end{center}
%  \end{block}
% \end{frame}
% 
% \begin{frame}{Simulation Flow}
%  \begin{block}{The Many Steps in Simulating a FinFET}
%   \begin{center}
%    \includegraphics[width=0.95\textwidth]{flow-laststep.png}
%   \end{center}
%  \end{block}
% \end{frame}

\begin{frame}{Simulation Flow}
 \begin{block}{The Many Steps in Simulating a FinFET}
  \begin{center}
   \includegraphics[width=0.95\textwidth]{flow-viennacl.png}
  \end{center}
 \end{block}
\end{frame}



%
% Introduce CG and perform first optimizations
%


%%
%% Conjugate Gradients: Pipelining
%%

% Show CG algorithm <-> BLAS


\begin{frame}[fragile]{Performance Modeling: Conjugate Gradients}

 \begin{block}{}
  
   \begin{minipage}{0.45\textwidth}
      {\large \textbf{Pseudocode}} \\
      
      Choose $x_0$ \\
      $p_0 = r_0 = b - Ax_0$ \\
      For $i=0$ until convergence
     \begin{enumerate}
      \item Compute and store $Ap_i$
      \item Compute $\langle p_i, Ap_i \rangle$
      \item $\alpha_i = \langle r_i, r_i \rangle / \langle p_i, Ap_i \rangle$
      \item $x_{i+1} = x_{i} + \alpha_i p_i$          
      \item $r_{i+1} = r_i - \alpha_i Ap_i$       
      \item Compute $\langle r_{i+1}, r_{i+1} \rangle$
      \item $\beta_i = \langle r_{i+1}, r_{i+1} \rangle / \langle r_i, r_i \rangle$
      \item $p_{i+1} = r_{i+1} + \beta_i p_i$
     \end{enumerate}
     EndFor
   \end{minipage}
   \begin{minipage}{0.48\textwidth}
      {\large \textbf{BLAS-based Implementation}} \\
      
            - \\
      SpMV, AXPY \\
      For $i=0$ until convergence
     \begin{enumerate}
      \item SpMV {\color{blue} $\leftarrow$ No caching of $Ap_i$}
      \item DOT {\color{red} $\leftarrow$ Global sync!}
      \item -
      \item AXPY         
      \item AXPY  {\color{blue} $\leftarrow$ No caching of $r_{i+1}$}
      \item DOT {\color{red} $\leftarrow$ Global sync!}
      \item -
      \item AXPY
     \end{enumerate}
     EndFor
   \end{minipage}
   
   \end{block}
   
\end{frame}

\begin{frame}[fragile]{Performance Modeling: Conjugate Gradients}

 \begin{block}{}
 
 \begin{center}
  \vspace*{-0.5cm}
  \includegraphics[width=0.85\textwidth]{figures/cg-k20m-0}
 \end{center}

 \begin{itemize}
  \item   \vspace*{-0.3cm} {\small (Poisson, 2D, Finite Differences)}
 \end{itemize}

 \end{block}
   
\end{frame}


\begin{frame}[fragile]{Performance Modeling: Conjugate Gradients}

 \begin{block}{Performance Modelling}
   \begin{itemize}
    \item 6 Kernel Launches (plus two for reductions)
    \item Two device to host data reads from dot products
    \item Model SpMV as seven vector accesses (5-point stencil)
    \item $T(N) = 8 \times 10^{-6} + 2 \times 2 \times 10^{-6} + (7+2+3+3+2+3) \times 8 \times N / \mathrm{Bandwidth}$
   \end{itemize}

 %\pause
 \begin{center}
  \vspace*{-0.2cm}
  \includegraphics[width=0.65\textwidth]{figures/cg-k20m-1}
 \end{center}

%  \begin{itemize}
%   \item   \vspace*{-0.5cm} {\small (Poisson, 2D, Finite Differences)}
%  \end{itemize}

 \end{block}
   
\end{frame}


%%%%%%%%

% Step 3: Show and discuss pipelined/improved version

\begin{frame}[fragile]{Performance Modeling: Conjugate Gradient Optimizations}

 \begin{block}{Optimization: Rearrange the algorithm}
   \begin{itemize}
   \item  Remove unnecessary reads 
   \item  Remove unnecessary synchronizations
   \item Use custom kernels instead of standard BLAS
  \end{itemize}
 \end{block}
   
\end{frame}



\begin{frame}[fragile]{Performance Modeling: Conjugate Gradients}

 \begin{block}{}
  
   \begin{minipage}{0.45\textwidth}
      {\large \textbf{Standard CG}} \\
      
      Choose $x_0$ \\
      $p_0 = r_0 = b - Ax_0$ \\
      For $i=0$ until convergence
     \begin{enumerate}
      \item Compute and store $Ap_i$
      \item Compute $\langle p_i, Ap_i \rangle$
      \item $\alpha_i = \langle r_i, r_i \rangle / \langle p_i, Ap_i \rangle$
      \item $x_{i+1} = x_{i} + \alpha_i p_i$          
      \item $r_{i+1} = r_i - \alpha_i Ap_i$       
      \item Compute $\langle r_{i+1}, r_{i+1} \rangle$
      \item $\beta_i = \langle r_{i+1}, r_{i+1} \rangle / \langle r_i, r_i \rangle$
      \item $p_{i+1} = r_{i+1} + \beta_i p_i$
     \end{enumerate}
     EndFor
   \end{minipage}
   \begin{minipage}{0.53\textwidth}
      {\large \textbf{Pipelined CG}} \\
      
      Choose $x_0$ \\
      $p_0 = r_0 = b - Ax_0$ \\
      For $i=1$ until convergence
     \begin{enumerate}
      \item $i=1$: Compute $\alpha_0$, $\beta_0$, $Ap_0$
      \item {\color{blue}$x_i = x_{i-1} + \alpha_{i-1} p_{i-1}$}
      \item {\color{blue}$r_i = r_{i-1} - \alpha_{i-1} Ap_i$}
      \item {\color{blue}$p_i = r_i + \beta_{i-1} p_{i-1}$}       
      \item {\color{red} Compute and store $Ap_i$}
      \item  {\color{red} Compute $\langle Ap_i, Ap_i \rangle$, $\langle p_i, Ap_i \rangle$}, {\color{blue}$\langle r_i, r_i \rangle$}
      \item $\alpha_i = \langle r_i, r_i \rangle / \langle p_i, Ap_i \rangle$
      \item $\beta_i = ( \alpha_i^2 \langle Ap_i, Ap_i \rangle - \langle r_i, r_i \rangle) / \langle r_i, r_i \rangle$
     \end{enumerate}
     EndFor
   \end{minipage}
   
   \end{block}
   
\end{frame}


\begin{frame}[fragile]{Performance Modeling: Conjugate Gradients}
 \begin{block}{}
 \begin{center}
  \vspace*{-0.5cm}
  \includegraphics[width=0.85\textwidth]{figures/cg-k20m-3}
 \end{center}

 \begin{itemize}
  \item   \vspace*{-0.3cm} {\small (Poisson, 2D, Finite Differences)}
 \end{itemize}
 \end{block}   
\end{frame}



\begin{frame}[fragile]{Performance Modeling: Conjugate Gradients}
 \begin{block}{Benefits of Pipelining also for Large Matrices}
 \begin{center}
  \vspace*{-0.2cm}
  \hspace*{-1.5cm}\includegraphics[width=1.05\textwidth]{figures/cg}
 \end{center}
  \vspace*{0.2cm}

 \end{block}   
\end{frame}




%
% Parameter Tuning: Explain benchmark setting, then present results
%


\begin{frame}{Benchmark Setting}

  \begin{block}{Scope for Portability Study}
    \begin{itemize}
     \item Vector and matrix-vector operations (BLAS levels 1 and 2)
     \item Limited by memory bandwidth
     %\item Matrix-matrix-multiplication (only briefly today)
    \end{itemize}
  \end{block}

  \pause
  \begin{block}{Key Question (Memory-Bandwidth-Limited Kernels)}
    \begin{center} \color{red} \LARGE
     Good performance of complicated kernels \\
     by optimizing the simplest kernel?
    \end{center}
  \end{block}

\end{frame}


%% Copy kernel:
\begin{frame}[fragile]{Benchmark Setting}
  \begin{block}{Vector Assignment (Copy) Kernel}
    \begin{itemize}
     \item $x \Leftarrow y$ for (large) vectors $x$, $y$
    \end{itemize}
  \end{block}

  \only<1>{\vspace*{4.22cm}}
  \only<2>{\begin{center} \vspace*{0.84cm} \includegraphics[width=0.8\textwidth]{figures/copy-kernel-gpu-1} \end{center}}
  \only<3>{\begin{center} \vspace*{0.85cm} \includegraphics[width=0.8\textwidth]{figures/copy-kernel-gpu-1} \end{center}}
  \only<4>{\begin{center}                  \includegraphics[width=0.8\textwidth]{figures/copy-kernel-gpu-full} \end{center}}
  \only<5>{\begin{center} \vspace*{0.84cm} \includegraphics[width=0.8\textwidth]{figures/copy-kernel-cpu-1} \end{center}}
  \only<6>{\begin{center} \vspace*{0.30cm} \includegraphics[width=0.8\textwidth]{figures/copy-kernel-cpu-full} \end{center}}
  
  \only<1>{\vspace*{2.52cm}}
  \only<2>{\vspace*{2.52cm}}
  \only<3>{
  \begin{block}{Parameters (1900 variations) }
   \begin{itemize}
    \item Local work size, global work size
    \item Vector types (float1, float2, ... , float16)
    \item Thread increment type
   \end{itemize}
  \end{block}}
  
  \only<4>{
  \begin{block}{Parameters (1900 variations) }
   \begin{itemize}
     \item \texttt{for (size\_t i = get\_global\_id(0); i < N;}
     \item \texttt{\ \ \ \ \ \ \ \ \ \ \ \ i+= get\_global\_size(0))}
     \item \texttt{\ \ x[i] = y[i];}
   \end{itemize}
  \end{block}
  }

  \only<5>{
  \begin{block}{Parameters (1900 variations) }
   \begin{itemize}
     \item \texttt{for (size\_t i = group\_start + get\_local\_id(0);}
     \item \texttt{\ \ \ \ \ \ \ \ \ \ \ \ i < group\_end;}
     \item \texttt{\ \ i+= get\_local\_size(0)) x[i] = y[i];}
   \end{itemize}
  \end{block}
  }

  \only<6>{
  \begin{block}{Parameters (1900 variations) }
   \begin{itemize}
     \item \texttt{for (size\_t i = group\_start + get\_local\_id(0);}
     \item \texttt{\ \ \ \ \ i < group\_end; i+= get\_local\_size(0)) }
     \item \texttt{\ \  x[i] = y[i];}
   \end{itemize}
  \end{block}
  }
\end{frame}


%% 
\begin{frame}{Benchmark Setting}

  \begin{block}{Operations}
   \begin{itemize}
    \item Vector copy, vector addition, inner product
    \item Matrix-vector product
   \end{itemize}
  \end{block}

  \only<1>{\begin{center} \includegraphics[width=0.8\textwidth]{figures/addition-kernel} \end{center}}
  \only<2>{\begin{center} \includegraphics[width=0.8\textwidth]{figures/inner-product-kernel} \end{center}}
  \only<3>{\begin{center} \includegraphics[width=0.8\textwidth]{figures/inner-product-kernel} \end{center}}

  \visible<3->{
  \begin{block}{Devices}
   \begin{itemize}
    \item AMD: A10-5800 APU, HD 5850 GPU
    \item INTEL: Dual Socket Xeon E5-2670, Xeon Phi
    \item NVIDIA: GTX 285, Tesla K20m
   \end{itemize}
  \end{block}
  }
  
\end{frame}



\begin{frame}{Portable Performance}
  \begin{center} \Large \textbf{Histograms} \end{center}
\end{frame}

\begin{frame}{Portable Performance}
  \begin{center} \includegraphics[width=0.75\textwidth]{figures/firepro_w9000_double_hist_itertype_dot} \end{center}
\end{frame}

\begin{frame}{Portable Performance}
  \begin{center} \includegraphics[width=0.75\textwidth]{figures/k20m_double_hist_itertype_dot} \end{center}
\end{frame}

\begin{frame}{Portable Performance}
  \begin{center} \includegraphics[width=0.75\textwidth]{figures/xeon_cpu_double_hist_itertype_dot} \end{center}
\end{frame}

\begin{frame}{Portable Performance}
  \begin{center} \includegraphics[width=0.55\textwidth]{figures/xeon_cpu_double_hist_itertype_dot} \\[1em]
                 \includegraphics[width=0.45\textwidth]{figures/copy-kernel-cpu-full} 
  \end{center}
\end{frame}


%%%%%%%%%%%%%%%%%%%%%%%%%%%%%%%%%%%%%%%%%%%%%%%%%%%%%%%%%%%%%%%


\begin{frame}{Portable Performance}
  \begin{center} \textbf{ [Addition$|$Inner Product$|$Matrix-Vector] vs. Copy Kernel }\\[1em] Same Device \end{center}
\end{frame}

\begin{frame}{Portable Performance}
  \only<1>{\begin{center} \includegraphics[width=0.8\textwidth]{figures/gtx285-addition-copy-1} \end{center}}
  \only<2>{\begin{center} \includegraphics[width=0.8\textwidth]{figures/gtx285-addition-copy-2} \end{center}}
  \only<3>{\begin{center} \includegraphics[width=0.8\textwidth]{figures/gtx285-addition-copy-3} \end{center}}
\end{frame}


\begin{frame}{Portable Performance}
  \only<1>{\begin{center} {\LARGE NVIDIA Tesla K20m} \\[2em]       \hspace*{1.cm} Addition \hspace*{2.1cm} Inner Product \hspace*{1.4cm} Mat-Vec Product \\[1em] \includegraphics[width=0.99\textwidth]{figures/k20m_double_xy_copy-crop} \end{center}}
  \only<2>{\begin{center} {\LARGE AMD Radeon HD 5850} \\[2em]      \hspace*{1.cm} Addition \hspace*{2.1cm} Inner Product \hspace*{1.4cm} Mat-Vec Product \\[1em] \includegraphics[width=0.99\textwidth]{figures/hd5850_double_xy_copy-crop} \end{center}}
  \only<3>{\begin{center} {\LARGE AMD FirePro W9000} \\[2em]       \hspace*{1.cm} Addition \hspace*{2.1cm} Inner Product \hspace*{1.4cm} Mat-Vec Product \\[1em] \includegraphics[width=0.99\textwidth]{figures/firepro_w9000_double_xy_copy} \end{center}}
  \only<4>{\begin{center} {\LARGE INTEL Dual Xeon E5-2670} \\[2em] \hspace*{1.cm} Addition \hspace*{2.1cm} Inner Product \hspace*{1.4cm} Mat-Vec Product \\[1em] \includegraphics[width=0.99\textwidth]{figures/xeon_cpu_double_xy_copy-crop} \end{center}}
  \only<5>{\begin{center} {\LARGE INTEL Xeon Phi} \\[2em]          \hspace*{1.cm} Addition \hspace*{2.1cm} Inner Product \hspace*{1.4cm} Mat-Vec Product \\[1em] \includegraphics[width=0.99\textwidth]{figures/xeon_phi_float_xy_copy-crop} \end{center}}
\end{frame}

\begin{frame}{Portable Performance}
  \begin{center} Conclusio: \\[1em]
    {\Large Focus on fastest configurations for copy-kernel sufficient} \\[5em]
    
    Good choice: \\[1em]
    {\Large Local workgroup size of 128 with 128 workgroups}
  \end{center}
\end{frame}



%%%%%%%%%%%%%%%%%%%%%%%%%%%%%%%%%%%%%%%%%%%%%%%%%%%%%%%%%%%%%%%



\begin{frame}{Portable Performance}
  \begin{block}{Matrix-Matrix Multiplication}
    \begin{itemize}
     \item Compute-bound
     \item Block-decomposition to maximize cache utilization
    \end{itemize}
  \end{block}
  
  \begin{center}
   \includegraphics[width=0.99\textwidth]{figures/MatrixMatrixProduct}
  \end{center}


\end{frame}


\begin{frame}{Portable Performance}
  \begin{center}
  Matrix-Matrix Multiplication (Single Precision) \\
  \includegraphics[width=0.8\textwidth]{figures/sgemm}
  \end{center} 
  
  {\small Ph.~Tillet \textit{et al.}: HotPar'13}
\end{frame}



%
% Continue with CG-method, show results for pipelined variant
%


% Step 3: Show and discuss pipelined/improved version

\begin{frame}[fragile]{Conjugate Gradient Optimizations}

 \begin{block}{Optimization: Rearrange the algorithm}
   \begin{itemize}
   \item  Remove unnecessary reads 
   \item  Remove unnecessary synchronizations
   \item Use custom kernels instead of standard BLAS
  \end{itemize}
 \end{block}
   
\end{frame}


\begin{frame}[fragile]{Conjugate Gradients}

 \begin{block}{}
  
   \begin{minipage}{0.45\textwidth}
      {\large \textbf{Standard CG}} \\
      
      Choose $x_0$ \\
      $p_0 = r_0 = b - Ax_0$ \\
      For $i=0$ until convergence
     \begin{enumerate}
      \item Compute and store $Ap_i$
      \item Compute $\langle p_i, Ap_i \rangle$
      \item $\alpha_i = \langle r_i, r_i \rangle / \langle p_i, Ap_i \rangle$
      \item $x_{i+1} = x_{i} + \alpha_i p_i$          
      \item $r_{i+1} = r_i - \alpha_i Ap_i$       
      \item Compute $\langle r_{i+1}, r_{i+1} \rangle$
      \item $\beta_i = \langle r_{i+1}, r_{i+1} \rangle / \langle r_i, r_i \rangle$
      \item $p_{i+1} = r_{i+1} + \beta_i p_i$
     \end{enumerate}
     EndFor
   \end{minipage}
   \begin{minipage}{0.53\textwidth}
      {\large \textbf{Pipelined CG}} \\
      
      Choose $x_0$ \\
      $p_0 = r_0 = b - Ax_0$ \\
      For $i=1$ until convergence
     \begin{enumerate}
      \item $i=1$: Compute $\alpha_0$, $\beta_0$, $Ap_0$
      \item {\color{blue}$x_i = x_{i-1} + \alpha_{i-1} p_{i-1}$}
      \item {\color{blue}$r_i = r_{i-1} - \alpha_{i-1} Ap_i$}
      \item {\color{blue}$p_i = r_i + \beta_{i-1} p_{i-1}$}       
      \item {\color{red} Compute and store $Ap_i$}
      \item  {\color{red} Compute $\langle Ap_i, Ap_i \rangle$, $\langle p_i, Ap_i \rangle$}, {\color{blue}$\langle r_i, r_i \rangle$}
      \item $\alpha_i = \langle r_i, r_i \rangle / \langle p_i, Ap_i \rangle$
      \item $\beta_i = ( \alpha_i^2 \langle Ap_i, Ap_i \rangle - \langle r_i, r_i \rangle) / \langle r_i, r_i \rangle$
     \end{enumerate}
     EndFor
   \end{minipage}
   
   \end{block}
   
\end{frame}



\begin{frame}[fragile]{Conjugate Gradients}
 \begin{block}{}
 \begin{center}
  \vspace*{-0.5cm}
  \includegraphics[width=0.8\textwidth]{figures/cg-k20m-3-crop}
 \end{center}

 \begin{itemize}
  \item   \vspace*{-0.3cm} {\small (2D Finite Difference Discretization)}
 \end{itemize}
 \end{block}   
\end{frame}


%%%%%%%%%

%\begin{frame}[fragile]{Conjugate Gradients}
% \begin{block}{Time per CG Iteration - NVIDIA K20m}
% \begin{center}
%  \vspace*{-0.01cm}
%  \includegraphics[width=0.75\textwidth]{figures/time-laplace2d-K20m-cg-1-crop}
% \end{center}

% \begin{itemize}
%  \item   \vspace*{-0.3cm} {\small (Laplace 2D Finite Element Discretization, Unstructured)}
% \end{itemize}
% \end{block}   
%\end{frame}

%\begin{frame}[fragile]{Conjugate Gradients}
% \begin{block}{Time per CG Iteration - NVIDIA K20m}
% \begin{center}
%  \vspace*{-0.01cm}
%  \includegraphics[width=0.75\textwidth]{figures/time-laplace2d-K20m-cg-2-crop}
% \end{center}

% \begin{itemize}
%  \item   \vspace*{-0.3cm} {\small (Laplace 2D Finite Element Discretization, Unstructured)}
% \end{itemize}
% \end{block}   
%\end{frame}

\begin{frame}[fragile]{Conjugate Gradients}
 \begin{block}{Time per CG Iteration - NVIDIA K20m}
 \begin{center}
  \vspace*{-0.01cm}
  \includegraphics[width=0.75\textwidth]{figures/time-laplace2d-K20m-cg-3-crop}
 \end{center}

 \begin{itemize}
  \item   \vspace*{-0.3cm} {\small (Laplace 2D Finite Element Discretization, Unstructured)}
 \end{itemize}
 \end{block}   
\end{frame}

%%%%

%\begin{frame}[fragile]{Conjugate Gradients}
% \begin{block}{Time per CG Iteration - NVIDIA K20m}
% \begin{center}
%  \vspace*{-0.2cm}
%  \includegraphics[width=0.75\textwidth]{figures/time-lame3d-K20m-cg-1-crop}
% \end{center}

% \begin{itemize}
%  \item   \vspace*{-0.3cm} {\small (3D Linear Elasticity, Linear Finite Elements, Unstructured)}
% \end{itemize}
% \end{block}   
%\end{frame}

%\begin{frame}[fragile]{Conjugate Gradients}
% \begin{block}{Time per CG Iteration - NVIDIA K20m}
% \begin{center}
%  \vspace*{-0.2cm}
%  \includegraphics[width=0.75\textwidth]{figures/time-lame3d-K20m-cg-2-crop}
% \end{center}

% \begin{itemize}
%  \item   \vspace*{-0.3cm} {\small (3D Linear Elasticity, Linear Finite Elements, Unstructured)}
% \end{itemize}
% \end{block}   
%\end{frame}

\begin{frame}[fragile]{Conjugate Gradients}
 \begin{block}{Time per CG Iteration - NVIDIA K20m}
 \begin{center}
  \vspace*{-0.2cm}
  \includegraphics[width=0.75\textwidth]{figures/time-lame3d-K20m-cg-3-crop}
 \end{center}

 \begin{itemize}
  \item   \vspace*{-0.3cm} {\small (3D Linear Elasticity, Linear Finite Elements, Unstructured)}
 \end{itemize}
 \end{block}   
\end{frame}


%%%%%%%% 

\begin{frame}[fragile]{Conjugate Gradients}
 \begin{block}{Time per CG Iteration - AMD FirePro W9100}
 \begin{center}
  \vspace*{-0.01cm}
  \includegraphics[width=0.75\textwidth]{figures/time-laplace2d-W9100-cg}
 \end{center}

 \begin{itemize}
  \item   \vspace*{-0.3cm} {\small (Laplace 2D Finite Element Discretization, Unstructured)}
 \end{itemize}
 \end{block}   
\end{frame}





%
% Discuss mdot implementation for use within GMRES:
%



\begin{frame}{Outline}
 \begin{center}
  \includegraphics[width=0.9\textwidth]{figures/outline-crop}
 \end{center}
\end{frame}


\begin{frame}[fragile]{BiCGStab and GMRES}
 \begin{block}{BiCGStab}
 \begin{itemize}
  \item Similar to CG
  \item Two SpMV per iteration
  \item Pipelining: 4 kernel launches instead of 12
 \end{itemize}
 \end{block}   
 
 \begin{block}{GMRES}
 \begin{itemize}
  \item Store Krylov basis
  \item Orthonormalization in each step
  \item Pipelining: 3 kernel launches
 \end{itemize}
 \end{block}   

 \begin{block}{Benchmark Setup}
 \begin{itemize}
  \item Poisson equation in 2D
  \item GPUs from NVIDIA and AMD
 \end{itemize}
 \end{block}   

\end{frame}



\begin{frame}[fragile]{BiCGStab Benchmarks}
 \begin{block}{}
 \begin{center}
  \includegraphics[width=0.7\textwidth]{figures/time-laplace2d-K20m-bicgstab}
 \end{center}
 \end{block}   
\end{frame}

\begin{frame}[fragile]{BiCGStab Benchmarks}
 \begin{block}{}
 \begin{center}
  \vspace*{-1cm}
  \includegraphics[width=0.9\textwidth]{figures/bicgstab}
 \end{center}
 \end{block}   
\end{frame}

\begin{frame}[fragile]{GMRES Benchmarks}
 \begin{block}{}
 \begin{center}
  \includegraphics[width=0.7\textwidth]{figures/time-laplace2d-K20m-gmres}
 \end{center}
 \end{block}   
\end{frame}

\begin{frame}[fragile]{GMRES Benchmarks}
 \begin{block}{}
 \begin{center}
  \vspace*{-1cm}
  \includegraphics[width=0.9\textwidth]{figures/gmres}
 \end{center}
 \end{block}   
\end{frame}





%
% Conclusion:
%
\begin{frame}{Summary and Conclusion}

  \begin{block}{Conjugate Gradient Method}
   \begin{itemize}
    \item Careful choice of sparse matrix format
    \item Tune kernels to target device
    \item Minimize reads from global memory (kernel fusion, pipelining)
   \end{itemize}
  \end{block}

  %\pause
  \begin{block}{Generalized Minimum Residual Method (GMRES)}
   \begin{itemize}
    \item Minimizes reads from global memory (mdot kernel)
    \item Up to twice the performance of 'naive' implementations
   \end{itemize}
  \end{block}
  
  %\pause
  \begin{block}{Implications}
    \begin{itemize}
     \item Tune primarily with memory transfers in mind
     \item Prefer regular memory access patterns
     \item Use \emph{appropriate} vector data types
    \end{itemize}
  \end{block}

\end{frame}

