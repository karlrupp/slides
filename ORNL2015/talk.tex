% -------------------------------------------------------------------
\documentclass[usepdftitle=false,10pt]{beamer}

%encodings
\usepackage[utf8]{inputenc}
\usepackage[OT1]{fontenc}

%some useful packages
\usepackage[]{mathrsfs}
\usepackage[]{amssymb}
\usepackage[]{amsmath}
\usepackage[]{acronym}
\usepackage[]{listings}
\usepackage[]{xcolor}
\usepackage[]{graphicx}
\usepackage[]{textcomp} %for tilde

\graphicspath{{figures/}}           % in which folder all the figures are
\newcommand{\tn}   {\textnormal}

% table of contents, depth
\setcounter{tocdepth}{2}

% modify at will
%\setbeamercovered{transparent}
\setbeamercovered{invisible}

% Enable IuE-theme (if available)
%\usetheme[height=0.3cm,width=0.8cm,shadow=false]{iue}

\author[Karl Rupp]{\normalfont \underline{Karl Rupp}$^{1,2}$, Andreas Morhammer$^{1}$,\\ Tibor Grasser$^{1}$, Ansgar J\"ungel$^{2}$}

\institute[TU Wien]
{ \footnotesize
  $^1$ Institute for Microelectronics \\
  $^2$ Institute for Analysis and Scientific Computing \\
  TU Wien, Austria  
}

\title[ViennaSHE]{Scaling Deterministic Numerical Solutions \\ of the Boltzmann Transport Equation \\ on Heterogeneous Computing Platforms}

\date[October 20th, 2015]{\footnotesize Scalable Methods for Kinetic Equations \\ Oak Ridge National Laboratory \\ October 20th, 2015}


\hypersetup{pdftitle={Scaling Deterministic Numerical Solutions of the Boltzmann Transport Equation on Heterogeneous Computing Platforms},
            pdfauthor={Karl Rupp},
            pdfsubject={Scalable Methods for Kinetic Equations}}

\setbeamertemplate{blocks}[default]
%\setbeamercolor{block title}{bg=}
\setbeamercolor{block body}{bg=}

\begin{document}

% Title page
\begin{frame}[plain]
 \frametitle{~}
 \titlepage
\end{frame}


%\section*{Contents}

\begin{frame}{Contents}
  \begin{block}{The Spherical Harmonics Expansion Method}
   \begin{itemize}
    \item Unstructured grids
    \item Adaptive variable-order expansions
    \item Parallelization
    %\item Carrier-carrier scattering
   \end{itemize}
  \end{block}
  
  \begin{block}{Solution on Distributed Memory Machines}
   \begin{itemize}
    \item Preconditioner blueprints
    \item Node-level parallel ILU
    \item Alternatives
   \end{itemize}
  \end{block}
  
\end{frame}



\begin{frame} {Semiconductor Devices in 3D: FinFET} 
  
 \begin{center} 
   {\footnotesize Intel Trigate transistors} \\
   \includegraphics[width=0.7\textwidth]{intel-trigate}
 \end{center} 
{\tiny http://download.intel.com/newsroom/kits/22nm/gallery/images/Intel-22nm\_Transistor\_2.jpg} \\[.6em]
 
\end{frame} 


\begin{frame} {Semiconductor Devices in 3D: FinFET} 
  
 \begin{center} 
   \includegraphics[width=0.6\textwidth]{trigate-n-2}
 \end{center} 
 
\end{frame} 


\begin{frame} {Electron Density in a FinFET} 
  
 \begin{center} 
  Density of electrons at each point $\mathbf x$ \\ 
  \includegraphics[width=0.632\textwidth]{trigate-electrons-mod}
 \end{center} 
  \vspace*{2.68cm}
 
\end{frame} 
 

\begin{frame} {Electron Energy Distribution?} 
 \begin{center}Distribution of electrons with respect to energy at $\mathbf{x}$? \\ 
  \includegraphics[width=0.87\textwidth]{trigate-electrons-edf-3}
 \end{center} 
\end{frame} 
 

\begin{frame} {Electron Energy Distribution?} 
 
  \begin{block}{Macroscopic Transport Models} 
    \begin{itemize} 
     \item Invalid in deca-nanometer regime 
     \item ``Fitting'' only treats the symptoms, not the cause 
     \item Only averaged quantities of the carrier ensemble modeled 
    \end{itemize} 
  \end{block} 
 
 %\visible<2->{
  \vspace*{0.5cm}
  \begin{block}{Boltzmann Transport Equation (BTE)} 
    \begin{align*} \Large 
      \frac{\partial f}{\partial t} +  \mathbf{v}(\mathbf{k}) \cdot \nabla_{\mathbf{x}} f + \mathbf F(\mathbf x) \cdot \nabla_{\mathbf k} f = Q\{f\}  
    \end{align*} 
  \vspace*{0.1cm}
 
    \begin{itemize} 
      \item Best semi-classical description of carrier transport 
      \item Posed in a seven-dimensional $(\mathbf x, \mathbf k, t)$ space 
      \item Most popular solution method: Monte Carlo 
    \end{itemize} 
  \end{block} 
  \vspace*{0.5cm}
  %}
\end{frame} 
 


\begin{frame} {Electron Energy Distribution?}
 \begin{center}
  \includegraphics[width=0.95\textwidth]{she-monte-carlo-2}
 \end{center}
\end{frame}
 

\section{Spherical Harmonics Expansion Method}




\begin{frame}{Spherical Harmonics Expansion Method}

 \vspace*{-0.3cm}
  \begin{block}{Spherical Symmetries}
  \begin{itemize}
   \item Maxwell distribution of carriers at equilibrium
   \item Dispersion relation (Herring-Vogt transform, approx.)
  \end{itemize}
  \end{block}

 
     \vspace*{0.62cm}
  \begin{block}{Spherical Harmonics Expansion (SHE)}
     \vspace*{-0.5cm}
      { %\Large
       \begin{align*}
	f(\mathbf x, \mathbf k, t) \simeq \sum_{l = 0}^L \sum_{m=-l}^l f_{l,m}(\mathbf x, E, t) Y_{l,m}(\theta, \varphi)
      \end{align*}}
     \vspace*{-0.5cm}
    \begin{itemize}
     \item New unknowns: $f_{l,m}(\mathbf x, E, t)$
     \item Solution in five-dimensional $(\mathbf x, E, t)$-space
     \item S.-M.~Hong and C.~Jungemann, \textit{J Comput Electron} (2009): \\
           Fifth-order, three-dim.~$(\mathbf x, E)$-space, 26 GB memory, 12 hours
    \end{itemize}
   \end{block}
%     \vspace*{-0.4cm}


\end{frame}





%%
%%
%%  M E T H O D S
%%
%%


%
% Unstructured Grids
%



\section{Unstructured Grids}

\begin{frame}{Unstructured Grids}
%  \vspace*{-0.5cm}
 \begin{block}{Unstructured Grids}
  \begin{itemize}
   \item State-of-the-art in modern TCAD
   \item Only structured grids in publications on higher-order SHE in 2D  \\
         \hspace{0.3cm} {\footnotesize [S.-M.~Hong and C.~Jungemann (2008), S.-M.~Hong and C.~Jungemann (2009)] }
   \item Extension of discretization proposed by Hong and Jungemann
  \end{itemize}
 \end{block}

 \begin{center}
  \begin{minipage}{0.2\textwidth}
    4838 nodes
  \end{minipage}
  \begin{minipage}{0.55\textwidth}
    \includegraphics[width=0.95\textwidth]{tetsortho}\ \ \\
  \end{minipage}
  \begin{minipage}{0.2\textwidth}
    27456 nodes
  \end{minipage}
 \end{center}
\end{frame}



% \begin{frame}{Summary}
%   \includegraphics[width=0.49\textwidth]{summary-exec-1}
%   \includegraphics[width=0.49\textwidth]{summary-memory-1}
% \end{frame}

\begin{frame}{Summary}
 \includegraphics[width=0.49\textwidth]{summary-exec-2}
 \includegraphics[width=0.49\textwidth]{summary-memory-2}
\end{frame}






%
% Adaptive Variable-Order
%


\section{Adaptive Variable-Order SHE}



\begin{frame}{Adaptive Variable-Order SHE}

  \vspace*{-0.05cm}
  \begin{block}{Spherical Harmonics Expansion}
     \vspace*{-0.6cm}
      { \begin{align*}
	f(\mathbf x, \mathbf k, t) \simeq \sum_{l = 0}^L \sum_{m=-l}^l f_{l,m}(\mathbf x, E, t) Y_{l,m}(\theta, \varphi)
      \end{align*}}
     \vspace*{-0.6cm}
  \end{block}

  \begin{itemize}
   \item $(L+1)^2$ unknown functions $f_{l,m}(\mathbf x, E, t)$
   \item $L=0$ sufficient in equilibrium
   \item Higher-order expansions in active regions
   \item Therefore: Variable-order SHE:
      { \Large \begin{align*}
	f(\mathbf x_i, \mathbf k_n, t) \simeq \sum_{l = 0}^{L(\mathbf x_i, E_n)} \sum_{m=-l}^l f_{l,m}(\mathbf x_i, E_n, t) Y_{l,m}(\theta, \varphi)
      \end{align*}}
   \item How to choose $L(\mathbf x_i, E_n)$ in the simulation domain?
  \end{itemize}
     \vspace*{0.92cm}
\end{frame}





\begin{frame}{Adaptive Variable-Order SHE}
  
 \begin{block}{Motivation from Fourier series}

  \vspace*{0.6cm}
  \begin{minipage}{0.97\textwidth}
    \includegraphics[width=0.45\textwidth]{fourier_f1} \hspace*{0.25cm}
    \includegraphics[width=0.45\textwidth]{fourier_f2} \hspace*{0.25cm} \newline

    \includegraphics[width=0.45\textwidth]{fourier_coeff_1}\hspace*{0.25cm}
    \includegraphics[width=0.45\textwidth]{fourier_coeff_2}
  \end{minipage}
 \end{block}

\end{frame}




\begin{frame}{Adaptive Variable-Order SHE}

 \fcolorbox{white}{white}{
  \begin{minipage}{0.1\textwidth}
    $L=1$ \\ 
    \vspace*{3cm} \\
    $L=3$
  \end{minipage}


  \begin{minipage}{0.4\textwidth}
  \begin{center} Error indicator:\\
    \includegraphics[width=0.9\textwidth]{mosfet-indicator-1} \\
    \includegraphics[width=0.9\textwidth]{mosfet-indicator-3}
  \end{center}
  \end{minipage}
  \hspace{0.5cm}
  \begin{minipage}{0.4\textwidth}
  \begin{center} Expansion order:\\
    \includegraphics[width=0.9\textwidth]{mosfet-order-1} \\
    \includegraphics[width=0.9\textwidth]{mosfet-order-3}
    %\vspace*{1cm}
  \end{center}  
  \end{minipage}
  \hspace{2.5cm}
 } %end of fcolorbox

%   \vspace*{-0.5cm}
\end{frame}


\begin{frame}{Adaptive Variable-Order SHE}
  \vspace{-0.3cm}
  \begin{center}
   \includegraphics[width=0.75\textwidth]{mosfet-energy-error}
  \end{center}
  \vspace{-0.4cm}
  \begin{center}
   %\begin{itemize}
    %\item 
      $L=3$: $\mathbf{306\, 261}$ instead of $\mathbf{\hphantom{1}\, 476\, 061}$ unknowns (factor $\mathbf{1.5}$) \\
    %\item 
      $L=5$: $\mathbf{606\, 671}$ instead of $\mathbf{1\, 146\, 120}$ unknowns (factor $\mathbf{1.9}$)
   %\end{itemize}
  \end{center}
%   \vspace*{-0.5cm}
   
\end{frame}

% \begin{frame}{Summary}
%  \includegraphics[width=0.49\textwidth]{summary-exec-2}
%  \includegraphics[width=0.49\textwidth]{summary-memory-2}
% \end{frame}

\begin{frame}{Summary}
  \includegraphics[width=0.49\textwidth]{summary-exec-3}
  \includegraphics[width=0.49\textwidth]{summary-memory-3}
\end{frame}




%
% Parallelization
%


\section{Parallelization}


\begin{frame}{Parallelization}
 \begin{block}{Preconditioner for Iterative Linear Solvers}
  \begin{itemize}
   \item No fast general-purpose parallel preconditioner available
   \item Physics-based parallel block preconditioner developed
  \end{itemize}
  \begin{center}
   \includegraphics[width=0.8\textwidth]{electron-trajectory-2}
  \end{center}
 \end{block}
%  \vspace*{-0.4cm}

\end{frame}

%%%


\begin{frame}{Parallelization}
 \begin{block}{Scaling of Solution Variables}
  \begin{itemize}
   \item Exponential decay with energy: $f(E_i) \sim \exp(- \frac{E_i}{k_\mathrm{B} T})$
   \item Rescale unknowns: $\tilde{f}(E_i) = \exp( \frac{E_i}{k_\mathrm{B} T}) f(E_i)$
   \item New system: $\tilde{\mathbf A} \tilde{\mathbf x} = \mathbf A \mathbf D \mathbf D^{-1} \mathbf x = \mathbf b$
   \item Row normalization: $\hat{\mathbf A} \tilde{\mathbf x} = \mathbf P \tilde{\mathbf A} \tilde{\mathbf x} = \mathbf P \mathbf b$
  \end{itemize}
  \begin{center}
   \includegraphics[width=0.45\textwidth]{matrix-structure-1} \hfill
   \includegraphics[width=0.45\textwidth]{matrix-structure-2}
  \end{center}
 \end{block}
%  \vspace*{-0.4cm}

\end{frame}


%%%

\begin{frame}{Parallelization}
  \vspace{-1.25cm}
  \begin{center}
   Benchmark results for a FinFET (INTEL Core i7 960, NVIDIA GTX 580) \\
   \vspace*{0.5cm}
  \includegraphics[width=0.48\textwidth]{mosfet-precondtimes} \hspace*{0.2cm}
  \includegraphics[width=0.48\textwidth]{mosfet-solvertimes}
  \end{center}
\end{frame}

\begin{frame}{Parallelization}
  \begin{center}
   Benchmark results for a FinFET (INTEL Core i7 960, NVIDIA GTX 580) \\
   \vspace*{0.5cm}
   \includegraphics[width=0.7\textwidth]{mosfet-totaltimes}
  \end{center}
\end{frame}




%
% Results
%


\section{Results}


\begin{frame}{Results}
%  \vspace{-0.5cm}
  \begin{minipage}{0.99\textwidth}
   \begin{center}
    Electron Concentration (cm$^{-3}$) \\
    \includegraphics[width=0.31\textwidth]{trigate-n-1} \hspace{1cm}
    \includegraphics[width=0.31\textwidth]{trigate-n-2}
   \end{center}
  \end{minipage}
    
  \vspace{0.5cm}
  \fcolorbox{white}{white}{
  \begin{minipage}{0.99\textwidth}
   \begin{center}
    Avg.~Expansion Order  \\
    \includegraphics[width=0.31\textwidth]{trigate-avg-order-1} \hspace{1cm}
    \includegraphics[width=0.31\textwidth]{trigate-avg-order-2}
   \end{center}
  \end{minipage}}
%  \vspace*{-0.5cm}
\end{frame}



\begin{frame}{Results}
%  \vspace{-0.5cm}
  \begin{minipage}{0.99\textwidth}
   \begin{center}
    Electron Concentration (cm$^{-3}$) \\
    \includegraphics[width=0.51\textwidth]{trigate-electrons-mod}
   \end{center}
  \end{minipage}
    
  \vspace{0.5cm}
  \fcolorbox{white}{white}{
  \begin{minipage}{0.99\textwidth}
   \begin{center}
    Electron Density with Energy above 1eV (cm$^{-3}$)  \\
    \includegraphics[width=0.51\textwidth]{trigate-electrons1eV-mod.png}
   \end{center}
  \end{minipage}}
%  \vspace*{-0.5cm}
\end{frame}





%
% Summary
%

\section{Summary}


\begin{frame}{Summary for Shared Memory Machines}
  \includegraphics[width=0.49\textwidth]{summary-exec-all}
  \includegraphics[width=0.49\textwidth]{summary-memory-all}
\end{frame}

%%%%%%%%%%% Part 2: Outlook to distributed memory %%%%%%%%%%%%%%


\begin{frame}{Part 2}
 \begin{center}
  Current Work: \\
  Development of a Parallel Preconditioner \\ 
  for (Heterogeneous) Distributed Memory Machines
 \end{center} 
\end{frame}



\begin{frame}{Distributed SHE}

  \begin{center}
    \includegraphics[width=0.45\textwidth]{matrix-structure-2}
  \end{center}

  %\pause
  
  \begin{block}{Blueprint}
   \begin{itemize}
    \item Keep block-Jacobi based on total energies
    \item Map total energies to MPI ranks
    \item Requires fine-grained parallel preconditioner per block
   \end{itemize}
  \end{block}
  
\end{frame}


\begin{frame}{Parallel ILU}

  \begin{block}{General}
   \begin{itemize}
    \item Approximate factorization $\mathbf{A} \approx \mathbf{L} \mathbf{U}$
    \item Proposed by Chow and Patel (SISC, vol.~37(2)) for CPUs and MICs
    \item Available in ViennaCL for CUDA, OpenCL, OpenMP
   \end{itemize}
  \end{block}
  
  %\pause

  \begin{block}{Preconditioner Setup}
   \begin{itemize}
    \item Nonlinear parallel sweeps to obtain $l_{ij}$ and $u_{ij}$
    \item Massively parallel (one thread per row)
   \end{itemize}
  \end{block}

  %\pause
  
  \begin{block}{Preconditioner Application}
   \begin{itemize}
    \item Truncated Neumann series:
     \begin{align*} \mathbf{L}^{-1} \approx \sum_{k=0}^K (\mathbf{I} - \mathbf{L})^k, \quad \mathbf{U}^{-1} \approx \sum_{k=0}^K (\mathbf{I} - \mathbf{U})^k \end{align*}
    \item Exact triangular solves not necessary
   \end{itemize}
  \end{block}

\end{frame}


\begin{frame}{Parallel ILU}

  \begin{center}
    \includegraphics[width=0.95\textwidth]{ilu-3d}
  \end{center}
\end{frame}


\begin{frame}{Preconditioner Evaluation}

  \begin{block}{Resolution}
   \begin{itemize}
    \item Total energy spacing: approx. $10$ meV 
    \item Thus: Hundred MPI ranks per eV (slight load imbalance)
    \item Typical minimum range: $3$-$5$ eV
   \end{itemize}
  \end{block}

  %\pause
  
  \begin{block}{Granularity}
   \begin{itemize}
    \item Fine-grained: One thread per matrix row on each MPI rank
    \item Coarse-grained: One run per voltage bias
   \end{itemize}
  \end{block}

  %\pause
  
  \begin{block}{Evaluation}
   \begin{itemize}
    \item Preconditioner sufficient for typical TCAD workloads
    \item Ready for upcoming HPC hardware
    \item Spatial domain decomposition required for strong scaling limit
   \end{itemize}
  \end{block}

\end{frame}

\begin{frame}{Preconditioner Alternatives}

  \begin{center}
    \includegraphics[width=0.4\textwidth]{matrix-structure-2}
  \end{center}

  \begin{block}{Other Shared Memory Preconditioners}
   \begin{itemize}
    \item Algebraic multigrid?
    \item Polynomial preconditioners?
   \end{itemize}
  \end{block}

  %\pause
  
  \begin{block}{Block-Diagonal Inversion}
   \begin{itemize}
    \item Additive Schwarz (no overlap)
    \item Sparse direct solver for each block?
   \end{itemize}
  \end{block}

\end{frame}


\begin{frame}{Parallel AMG}
  \begin{center}
    \includegraphics[width=0.95\textwidth]{amg-vs-pure-full-3d}
  \end{center}
\end{frame}



%%

\begin{frame}{Conclusion}

  \begin{block}{SHE Method}
   \begin{itemize}
    \item Viable alternative to Monte Carlo
    \item Full 3D device simulations possible
    \item Convergence behavior similar to drift-diffusion model
    \item Free open-source simulator: ViennaSHE
   \end{itemize}
  \end{block}
  
  %\pause
  
  \begin{block}{Large-Scale Solution}
   \begin{itemize}
    \item Physics-based block-Jacobi preconditioner
    \item Replication of spatial mesh on all MPI ranks
    \item Fine-grained parallel ILU
    \item Combine functionality in PETSc and ViennaCL libraries
   \end{itemize}
  \end{block}
  
\end{frame}


\end{document}

