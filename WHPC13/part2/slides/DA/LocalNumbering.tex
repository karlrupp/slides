\begin{frame}{DMDA Global vs. Local Numbering}

\begin{itemize}
  \item {\bf Global}: Each vertex has a unique id, belongs on a unique process

  \item {\bf Local}: Numbering includes vertices from neighboring processes
  \begin{itemize}
    \item These are called {\color{cyan}ghost} vertices
  \end{itemize}
\end{itemize}

\begin{center}
\begin{tabular}{cc}
\begin{tabular}{c}
\begin{tabular}{|ccc|cc|}
\hline
\multicolumn{3}{|c|}{Proc 2} & \multicolumn{2}{c|}{Proc 3} \\
\hline
 X &  X &  X &  X &  X \\
 X &  X &  X &  X &  X \\
{\color{cyan}12} & {\color{cyan}13} & {\color{cyan}14} & {\color{cyan}15} &  X \\
\hline
 8 &  9 & 10 & {\color{cyan}11} &  X \\
 4 &  5 &  6 & {\color{cyan}7} &  X \\
 0 &  1 &  2 & {\color{cyan}3} &  X \\
\hline
\multicolumn{3}{|c|}{Proc 0} & \multicolumn{2}{c|}{Proc 1} \\
\hline
\end{tabular} \\
Local numbering
\end{tabular}
& 
\begin{tabular}{c}
\begin{tabular}{|ccc|cc|}
\hline
\multicolumn{3}{|c|}{Proc 2} & \multicolumn{2}{c|}{Proc 3} \\
\hline
21 & 22 & 23 & 28 & 29 \\
18 & 19 & 20 & 26 & 27 \\
15 & 16 & 17 & 24 & 25 \\
\hline
 6 &  7 &  8 & 13 & 14 \\
 3 &  4 &  5 & 11 & 12 \\
 0 &  1 &  2 &  9 & 10 \\
\hline
\multicolumn{3}{|c|}{Proc 0} & \multicolumn{2}{c|}{Proc 1} \\
\hline
\end{tabular}\\
Global numbering
\end{tabular}
\end{tabular}
\end{center}
\end{frame}
