\begin{frame}{Preconditioning}
  \begin{definition}[Preconditioner]
      A \emph{preconditioner} $\mathcal{P}$ is a method for constructing a matrix
      $P^{-1} = \mathcal{P}(A,A_p)$ using a matrix $A$ and extra information $A_p$, such that
      the spectrum of $P^{-1}A$ (or $A P^{-1}$) is well-behaved.
    \end{definition}
    \begin{itemize}
    \item $P^{-1}$ is dense, $P$ is often not available and is not needed
    \item $A$ is rarely used by $\mathcal{P}$, but $A_p = A$ is common
    \item $A_p$ is often a sparse matrix, the ``preconditioning matrix''
    \item Matrix-based: Jacobi, Gauss-Seidel, SOR, ILU(k), LU
    \item Parallel: Block-Jacobi, Schwarz, Multigrid, FETI-DP, BDDC
    \item Indefinite: Schur-complement, Domain Decomposition, Multigrid
    \end{itemize}
\end{frame}
