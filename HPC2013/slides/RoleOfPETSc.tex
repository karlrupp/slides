\begin{frame}[fragile]

\frametitle{The Role of PETSc}

\vspace*{\fill}
\begin{minipage}{\linewidth}
\begin{quote}
\Large Developing parallel, nontrivial PDE solvers that deliver high performance is still difficult and requires
months (or even years) of concentrated effort.

\medskip

PETSc is a toolkit that can ease these difficulties and reduce the development time, but it is not a black-box PDE
solver, nor a \color{blue}{silver bullet}.
\end{quote}
\qquad --- Barry Smith
\end{minipage}
\vspace*{\fill}\vspace*{\fill}

\end{frame}



\begin{frame}[fragile]

\frametitle{The Role of PETSc}

\vspace*{\fill}
\begin{minipage}{\linewidth}
\begin{quote}
\Large You want to think about how you decompose your data
structures, how you think about them globally. [...] 

\medskip

If you
were building a house, you'd start with a set of blueprints
that give you a picture of what the whole house looks
like. You wouldn’t start with a bunch of tiles and say.
``Well I'll put this tile down on the ground, and then I'll
find a tile to go next to it.''

\medskip

But all too many people try to
build their parallel programs by creating the smallest
possible tiles and then trying to have the structure of
their code emerge from the chaos of all these little
pieces. You have to have an organizing principle if
you're going to survive making your code parallel.

\end{quote}

\qquad --- Bill Gropp

\qquad --- http://www.rce-cast.com/Podcast/rce-28-mpich2.html
\end{minipage}
\vspace*{\fill}\vspace*{\fill}

\end{frame}
