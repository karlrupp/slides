


\begin{frame}{History of OpenCL}

\begin{minipage}{0.7\textwidth}
\begin{block}{Prior to 2008}
  \begin{itemize}
   \item OpenCL developed by Apple Inc.
  \end{itemize}
\end{block}

\begin{block}{2008}
  \begin{itemize}
   \item OpenCL working group formed at Khronos Group
   \item OpenCL specification 1.0 released
  \end{itemize}
\end{block}

\begin{block}{2010}
  \begin{itemize}
   \item OpenCL 1.1 (multi-device, subbuffer manipulation)
  \end{itemize}
\end{block}

\begin{block}{2011}
  \begin{itemize}
   \item OpenCL 1.2 (device partitioning)
  \end{itemize}
\end{block}

\begin{block}{2013}
  \begin{itemize}
   \item OpenCL 2.0 (shared virtual memory, SPIR, etc.)
  \end{itemize}
\end{block}

\end{minipage} \hfill
\begin{minipage}{0.25\textwidth}
 \includegraphics[width=0.99\textwidth]{figures/opencl.jpg}
 
 \vspace*{5cm}
\end{minipage}


\end{frame}





\begin{frame}{About OpenCL}

\begin{block}{Advantages}
  \begin{itemize}
   \item Not restricted to a single vendor: Intel, NVIDIA, AMD, ...
   \item Just a shared C-library
   \item Does not rely on compiler magic
  \end{itemize}
\end{block}

\pause
\begin{block}{Disadvantages}
  \begin{itemize}
   \item Not restricted to a single vendor
   \item Boilerplate code required
   \item Portable performance?
  \end{itemize}
\end{block}

\end{frame}


